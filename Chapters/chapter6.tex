In this thesis we presented the use of \textit{Score-Based Generative Models} \cite{score_1, score_2, score_3} for \textit{Semantic Image Synthesis}. Semantic image synthesis provides human control over the semantic composition of an image, making image generation more versatile and useful for image generation applications. Furthermore it enables image generation on datasets that would otherwise hard to learn by unconditional models. Semantic image synthesis with score-based generative models brings a number of benefits that makes it particularly desirable to use: Unlike standalone models such as CRN \cite{crn}, pix2pixHD \cite{pix2pixHD}, and SPADE \cite{spade}, which are explicitly designed for semantic image synthesis as the result of a research project, the approach presented in this thesis is not a model per se, but a modular system that can evolve over time and has many parts that can be adapted and improved independently. Any future advancement in noise-conditional semantic segmentation networks, score-based generative models or sampling techniques can be used for semantic image synthesis in an easy-to-implement plug-and-play fashion. After training an unconditional score model like NSCN++ \cite{score_3}, one can then decide how to use this model. With a few minor changes, the trained model can be used for colorization, class-conditional sampling, or semantic image synthesis, or presumably all at once.

In the conducted experiments (see \hyperref[sec:5.4]{Sec.\,5.4}, \hyperref[sec:5.5]{Sec.\,5.5}, \hyperref[sec:5.6]{Sec.\,5.6}), we show not only that semantic image synthesis with score-based generative models is generally possible, but also that we can achieve good results even with a very simple semantic segmentation model like U-Net \cite{unet}. The comparison of our samples on the Cityscapes dataset \cite{cityscapes} with those of the state-of-the-art models CRN, pix2pixHD and SPADE yielded promising results, although they could not beat the favorite SPADE. As a result for this experiment, our model reached the second best FID score \cite{fid} and the third place in pixel accuracy. However, our model was found to perform very poor on fine structures of the semantic map. This problem is due to the limited capability of the simple noise-conditional U-Net on high noise scales, which was confirmed on the ADE20K \cite{ade20k} dataset where the poor U-Net performance allows for no realistic semantic image synthesis. In turn, on a simpler dataset with landscape images scraped from Flickr with only few labels and coarse structures in the semantic maps, our approach is able to generative good-looking images on high resolutions up to $1024\times512$ pixels.

In conclusion we see two main issues that have to be tackled in future research to really make semantic image synthesis with score-based generative models a attractive state-of-the-art approach: First, one needs to find a semantic segmentation model that works better on high image noise. We suggest that such a model will most likely need to be developed specifically for this task, as our attempts to use more complex models such as FCDenseNet \cite{densenet} performed worse than U-Net. Also, most modern semantic segmentation models rely on pretrained models which obviously are not conditioned on noise. And second, and there already is a lot of research in this direction, the sampling time of score-based generative models should be decreased, as it is a large hindrance for parameter finetuning and does not allow for real time image generation tasks. In \cite{gotta_go_fast}, a promising new sampler was presented that can reduce sampling times by a factor of up to $10$. Another interesting approach is the use of score-based generative models in latent space, which was recently achieved in \cite{latent} and allows up to $600\times$ faster sampling speed than before. However, whether our approach is also applicable in latent space is an open question and left to future research.

Altogether, our approach combines the versatility and high sampling quality of score-based generative models with semantic image synthesis, making it a powerful framework for future research in this field.


